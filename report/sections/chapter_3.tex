
\section{Generation of the strings}

In order to make the experimentations easier the input strings are automaticaly generated using patterns.
Those patterns are put in a file by the user which allows him to quickly be able to experiment with different cases.
Then when the Enumeration class is instanciated it reads the patterns from the file and all the strings are generated and stocked in a vector, then the strings can be accessed easily from anywhere.

The patterns uses the following expression:
$$
c \string^ x \text{\textvisiblespace} d \string^ y
$$

Where $c$ is an ASCII character and $x$ is the number of time we add the character, a space separates each characters.
It is also possible not to renseignate $\string^ x$, then the character is added 1 time.

That method allows the user to generate new strings for different grammars very quickly.

\section{Experimentations}

In some of the experimentations the results of the naive parser will not be displayed, which means that it was not efficient enough to be interesting.

The results of the `boolean' bottom-up parser will be displayed only once per grammar since no matter the pattern of the input string it does not change the behaviour of this parser.
However its results might be used several times for a same grammar in order to use them as reference for the other parsers.

\subsection{Well balanced parentheses}
The grammar is the following:

\begin{align*} 
    &S \to SS|LA|LR\\
    &A \to SR\\
    &L \to (\\
    &R \to )
\end{align*}

With this grammar it is possible to demonstrate that the top-down parser needs less recursive calls when it can `detect' that the string does not match the grammar very quickly.

\subsubsection{Number of iterations/recursive calls for each parsers}

With this grammar two interesting cases will be considered:

\begin{enumerate}
    \item Any pattern starting with a right parenthese:
    $$
    patterns = 
    \begin{cases}
        )\string^ n\\
        ) \text{ } (\string^ n - 1\\
        )\string^ n - 1 \text{ } (\\
    \end{cases}
    $$
    Many other patterns corresponding to the same case could be found, as long as the first character is a right parenthese.
    This case is interesting because it is the case in which the top-down parser needs to solve the less sub-problems in order to parse the strings.
    
    For example if it parses a string of size $n$ based on one of the patterns above it will need to solve $2n - 1$ sub-problems to give the result.
    That information, while interesting, is not relevant to anticipate the running time of the parser since even if we know how many sub-problems the parser will solve it can need a very different amount of recursive calls.
    
    \item The other interesting case is the pattern `$(\string^ n$', that pattern is the worst that can be generated with this grammar for the top-down parser.

    With that pattern the top-down parser will need to solve $n^2 + \lfloor \dfrac{n}{2} \rfloor$ sub-problems to parse a string of size $n$.
\end{enumerate}

Of course as demonstrated in the section 2.4 with equation \ref{eq:bottom-up_iterations} the `boolean' bottom-up parser always needs the same amount of iterations for a string of size $n$ and a given grammar:

\begin{itemize}
    \item[$-$] $n = 500$
    \item[$-$] $gt = 2$
    \item[$-$] $gnt = 4$
\end{itemize}

$$
500 \cdot 2 + \dfrac{4}{6} \cdot (3 \cdot 500^2 \cdot (500 + 1) - 500 \cdot (500 + 1) \cdot (2 \cdot 500 + 1)) = \text{83,334,000 iterations}
$$

Which is indeed the number of iterations obtained when running the code.

The `string' bottom-up parser will also follow the same behaviour for any string pattern.

It is now possible to represent the anticiped behaviour that the `boolean' bottom-up parser will follow for both cases.

\FloatBarrier
\begin{figure}[h]
\centering
\begin{tikzpicture}
\begin{groupplot}[group style={group size=1 by 1},height=0.5\textwidth,width=0.5\textwidth, domain=0:1000]
    \nextgroupplot[title=`Boolean' bottom-up for both cases, ylabel=iterations, xlabel=string size]
    \addplot[red]{2 * x + ((4 / 6) * ((3 * (x^2) * (x + 1)) - (x * (x + 1) * ((2 * x) + 1))))};
\end{groupplot}
\end{tikzpicture}
\caption{Anticipation of the `boolean' bottom-up parser, grammar 1}
\end{figure}
\FloatBarrier

\subsubsection{Comparing the efficiency}

The first experimentation uses the case in which the strings are starting with a right patenthese: $)\string^ n$.

\FloatBarrier
\begin{figure}[h]
\begin{tikzpicture}
    \begin{groupplot}[group style={group size=2 by 2},height=0.5\textwidth,width=0.5\textwidth] 
    \nextgroupplot[title=Both bottom-up counters, ylabel=iterations, legend pos=north west]
    \addplot coordinates {
        (50, 21121)
        (100, 167246)
        (200, 1334496)
        (400, 10668996)
        (600, 36003496)
        (800, 85337996)
        (1000, 166672496)};
    \addlegendentry{String}
    \addplot coordinates {
        (50, 83400)
        (100, 666800)
        (200, 5333600)
        (400, 42667200)
        (600, 144000800)
        (800, 341334400)
        (1000, 666668000)};
    \addlegendentry{Boolean}
    \nextgroupplot[title=Both bottom-up running times, ylabel=seconds, legend pos=north west]
    \addplot coordinates {
        (50, 0.000571)
        (100, 0.003931)
        (200, 0.031064)
        (400, 0.354137)
        (600, 1.25128)
        (800, 3.00095)
        (1000, 6.0186)};
    \addlegendentry{String}
    \addplot coordinates {
        (50, 0.002598)
        (100, 0.019514)
        (200, 0.142831)
        (400, 1.17313)
        (600, 4.05125)
        (800, 10.2203)
        (1000, 20.8444)};
    \addlegendentry{Boolean}
    \nextgroupplot[title=Top-down counter, xlabel=string size, ylabel=recursive calls]
    \addplot coordinates {
        (50, 3676)
        (100, 14851)
        (200, 59701)
        (400, 239401)
        (600, 539101)
        (800, 958801)
        (1000, 1498501)};
    \nextgroupplot[title=Top-down running time, xlabel=string size, ylabel=seconds, legend pos=north west]
    \addplot coordinates {
        (50, 0.000117)
        (100, 0.000302)
        (200, 0.000982)
        (400, 0.003857)
        (600, 0.007999)
        (800, 0.015024)
        (1000, 0.022071)};
    \addlegendentry{Total}
    \addplot coordinates {
        (50, 0.000048)
        (100, 0.000097)
        (200, 0.000234)
        (400, 0.000962)
        (600, 0.00241)
        (800, 0.003652)
        (1000, 0.005607)};
    \addlegendentry{Initialization}
    \addplot coordinates {
        (50, 0.000046)
        (100, 0.000195)
        (200, 0.000733)
        (400, 0.002882)
        (600, 0.006131)
        (800, 0.011)
        (1000, 0.017119)};
    \addlegendentry{Parsing}
    \end{groupplot}
\end{tikzpicture}
\caption{Both bottom-up and top-down parsers behaviours, grammar 1, case 1}
\end{figure}
\FloatBarrier

The `boolean' bottom-up parser follows exactly the predicted behaviour, the number of iterations going up following the equation \ref{eq:bottom-up_iterations} and its running time follows exactly the same curve since for each iteration the parser does exactly the same amount of loops.
For this grammar it is slower than the `string' version.

Here the top-down parser is faster than both bottom-up parsers.
To get a more precise result the initialization and the parsing process have been timed apart from each other.
We can see that the initialization time scales up a power function.
The parsing time seems to follow the curve of the recursive calls counter.
\\
\\
It is very easy to check the complexity of the two bottom-up parsers here by resolving for both an equation of the form $y = a \cdot x^3$ by replacing $x$ and $y$ by the coordinates of a point of their graphs.

Those are respectively the theorical expressions of the `boolean' and `string' bottom-up parsers:

\begin{align*}
    &y = 2.0844 \cdot 10^{-8} \cdot x^3 &y = 6.0186 \cdot 10^{-9} \cdot x^3
\end{align*}

\FloatBarrier
\begin{figure}[h]
\begin{tikzpicture}
    \begin{groupplot}[group style={group size=2 by 1},height=0.5\textwidth,width=0.5\textwidth] 
    \nextgroupplot[title=`Boolean' bottom-up, xlabel=string size, ylabel=seconds, legend pos=north west]
    \addplot[domain=0:1000, samples=100, line width=1.5pt, green] {2.0844*(10^(-8))*(x^3)};
    \addlegendentry{Theory}
    \addplot[only marks, mark=*, mark size=2pt] coordinates {
        (50, 0.002598)
        (100, 0.019514)
        (200, 0.142831)
        (400, 1.17313)
        (600, 4.05125)
        (800, 10.2203)
        (1000, 20.8444)};
    \addlegendentry{Real}
    \nextgroupplot[title=`String' bottom-up, xlabel=string size, ylabel=seconds, legend pos=north west]
    \addplot[domain=0:1000, samples=100, line width=1.5pt, green] {6.0186*(10^(-9))*(x^3)};
    \addlegendentry{Theory}
    \addplot[only marks, mark=*, mark size=2pt] coordinates {
        (50, 0.000571)
        (100, 0.003931)
        (200, 0.031064)
        (400, 0.354137)
        (600, 1.25128)
        (800, 3.00095)
        (1000, 6.0186)};
    \addlegendentry{Real}
    \end{groupplot}
\end{tikzpicture}
\caption{Checking theorical fit, both bottom-up parsers}
\end{figure}
\FloatBarrier

The two functions fit almost perfectly to the points which confirms the $O(n^3)$ complexity for both bottom-up parsers.
\\
\\
The naive parser is very slow with that grammar, but it is possible to check its complexity too by resolving the equation $y = a \cdot 3^x$.

\begin{align*}
    &y = 2.146422 \cdot 10^{-10} \cdot 3^x
\end{align*}

\FloatBarrier
\begin{figure}[h]
\centering
\begin{tikzpicture}
    \begin{groupplot}[group style={group size=1 by 1},height=0.5\textwidth,width=0.5\textwidth] 
    \nextgroupplot[title=naive, xlabel=string size, ylabel=seconds, legend pos=north west]
    \addplot[domain=10:23, samples=100, line width=1.5pt, green] {2.146422*(10^(-10))*(3^x)};
    \addlegendentry{Theory}
    \addplot[only marks, mark=*, mark size=2pt] coordinates {
        (10, 0.000236)
        (15, 0.017009)
        (18, 0.248498)
        (20, 1.39282)
        (21, 3.36746)
        (22, 7.98121)
        (23, 20.2071)};
    \addlegendentry{Real}
    \end{groupplot}
\end{tikzpicture}
\caption{Checking theorical fit, naive parser}
\end{figure}
\FloatBarrier

The theorical curve fits perfectly the running time of the naive parser which proves that the complexity of the naive parser is $O(3^n)$, it also shows that with that grammar this parser is very bad compared to the others.
\\
\\
The second pattern is the pattern only constituated by left parentheses: `$(\string^ n$', what makes it interesting is the fact that it is the worst case that we can generate with this grammar for the top-down parser.
The bottom-up parsers results will not be displayed in that case since they have the same behaviour as with the first case.

\FloatBarrier
\begin{figure}[h]
\begin{tikzpicture}
    \begin{groupplot}[group style={group size=2 by 1},height=0.5\textwidth,width=0.5\textwidth] 
    \nextgroupplot[title=Top-down counter, xlabel=string size, ylabel=recursive calls]
    \addplot coordinates {
        (50, 43526)
        (100, 340801)
        (200, 2696601)
        (400, 21453201)
        (600, 72269801)
        (800, 171146401)
        (1000, 334083001)};
    \nextgroupplot[title=Top-down running time, xlabel=string size, ylabel=seconds, legend pos=north west]
    \addplot coordinates {
        (50, 0.000605)
        (100, 0.004426)
        (200, 0.031748)
        (400, 0.252762)
        (600, 0.819545)
        (800, 1.88128)
        (1000, 3.71316)};
    \addlegendentry{Total}
    \addplot coordinates {
        (50, 0.000033)
        (100, 0.000119)
        (200, 0.000244)
        (400, 0.00081)
        (600, 0.001904)
        (800, 0.003088)
        (1000, 0.005164)};
    \addlegendentry{Initialization}
    \addplot coordinates {
        (50, 0.000561)
        (100, 0.004297)
        (200, 0.031488)
        (400, 0.251933)
        (600, 0.817614)
        (800, 1.87816)
        (1000, 3.70797)};
    \addlegendentry{Parsing}
    \end{groupplot}
\end{tikzpicture}
\caption{Top-down parser behaviour, grammar 1, case 2}
\end{figure}
\FloatBarrier

As we can see on the graph above in that case the initialization time of the top-down parser is negligible, the parsing time is the one that matters.
The parsing time follows exactly the same curve as the number of recursive calls.
Despite the fact that this case is the worst one for the top-down parser its running time remains lower than the other parsers.

We can conclude that for this grammar the top-down parser is more efficient than both bottom-up algorithms.
\\
\\
Here we can verify the complexity of the top-down parser by resolving the same equation as for the two bottom-up parsers.

\begin{align*}
    &y = 3.10797 \cdot 10^{-9} \cdot x^3
\end{align*}

\FloatBarrier
\begin{figure}[h]
\centering
\begin{tikzpicture}
    \begin{groupplot}[group style={group size=1 by 1},height=0.5\textwidth,width=0.5\textwidth] 
    \nextgroupplot[title=top-down, xlabel=string size, ylabel=seconds, legend pos=north west]
    \addplot[domain=0:1000, samples=100, line width=1.5pt, green] {3.70797*(10^(-9))*(x^3)};
    \addlegendentry{Theory}
    \addplot[only marks, mark=*, mark size=2pt] coordinates {
        (50, 0.000561)
        (100, 0.004297)
        (200, 0.031488)
        (400, 0.251933)
        (600, 0.817614)
        (800, 1.87816)
        (1000, 3.70797)};
    \addlegendentry{Real}
    \end{groupplot}
\end{tikzpicture}
\caption{Checking theorical fit, grammar 1, case 2}
\end{figure}
\FloatBarrier

The curve fits perfectly the points of the top-down parser which proves that the complexity of the algorithm is $O(n^3)$.



\subsection{String starting with an `a'}
The grammar used in this section is the following:

\begin{align*} 
    &S \to AB\\
    &A \to a\\
    &B \to BB|a|b\\
\end{align*}

This grammar matches every string that begins with an `a', no matter what comes next.

\subsubsection{Number of iterations/recusive calls for each parser}

It is very interesting because with this grammar the top-down algorithm never uses the stored values in its table, no matter the pattern used.
At first one could think that this is a very bad thing, indeed since the top-down without the table is exactly the same as the naive parser it means that using this grammar they will both have the same behaviour.
But what is interesting is that the naive parser with this grammar and a string of size $n$ reacts like so:
$$
recursive \text{ } calls = 
\begin{cases}
    2n - 1 &\text{if string matches the grammar}\\
    n &\text{otherwise}
\end{cases}
$$

It is very easy to demonstrate those two results:
\begin{enumerate}
    \item In every pattern that does start with an `a' the number of recursive calls is $2n - 1$ because:
        \begin{itemize}
            \item[$-$] The initial call is $parse(S, 0, n)$.
            \item[$-$] The function will the recursively call $parse(A, 0, 1)$, which will return true since the \textit{terminal} `a' is equal to the character 0 of the string which is also `a'.
            \item[$-$] Then the function will call a second time $parse(B, 1, n)$.
            \item[$-$] That function will call $parse(B, 1, 2)$. That will return true no matter the character since `B' possesses both \textit{terminals} `a' and `b'.
            \item[$-$] Then it will call $parse(B, 2, n)$. This will repeat itself until $parse(B, n - 1, n)$ is called. At that moment it will return true.
            \item[$-$] At that moment the number of recursive calls is $(n - 1) * 2$ plus the initial call $1$, for a total of $2n - 1$ recursive calls. Note that every 2 steps this algorithm goes 1 step deeper into the recursive stack, this is an important fact for the experimentations.
        \end{itemize}
    \item In every pattern that does not start with an `a' the number of recursive calls is the size of the string $n$ because:
        \begin{itemize}
            \item[$-$] The initial call is $parse(S, 0, n)$. That function will enter in a loop for k going from $1$ to $n - 1$.
            \item[$-$] The function will then recursively call $parse(A, 0, 1)$, which returns false since the \textit{terminal} `a' is not equal to the character 0 of the string, which is `b'.
            \item[$-$] Then the function will call $parse(A, 0, k)$ until the end of the for loop. The calls will return false each time since `A' does not possess any \textit{non-terminals} rules.
            \item[$-$] So there is $1$ call plus $n - 1$ calls made by the first called function, for a total of $n$ recursive calls. Note that all the sub calls are made by the first called function, which limits the maximum reached recursive depth to only 2, it will be interesting to see in the experimentations how that has a very important impact.
        \end{itemize}
\end{enumerate}

With a behaviour like this one the naive parser is very efficient with that grammar.
Also in this section only the naive parser and both bottom-up parsers will be considered, since the naive and top-down one are the same.

With this grammar the best and worst case scenarios for the `string' bottom-up parser are:
\begin{itemize}
    \item[$-$] The worst case is a string following this pattern: `$b\string^ n$'.
    \item[$-$] The best case is a string following this pattern: `$a\string^ n$'.
    \item[$-$] Every case that is a mix between `a's and `b's will have a running time situated between the best and worst case.
        The more `b's there is in the string the closer the running time will be from the worst case and conversely with `a's.
\end{itemize}

It is now poosible to compare the anticipated behaviours of the `boolean' bottom-up and the naive parser, in both cases, using the function \ref{eq:bottom-up_iterations} and the two expressions from above.

\FloatBarrier
\begin{figure}[h]
\begin{tikzpicture}
\begin{groupplot}[group style={group size=2 by 1},height=0.5\textwidth,width=0.5\textwidth, domain=0:1000]
\nextgroupplot[title=`Boolean' bottom-up, xlabel=string size, ylabel=iterations]
\addplot[red]{3 * x + ((2 / 6) * ((3 * (x^2) * (x + 1)) - (x * (x + 1) * ((2 * x) + 1))))};
\nextgroupplot[title=Naive, legend pos=north west,xlabel=string size, ylabel=recusive calls]
\addplot[green]{2 * x - 1};
\addlegendentry{Case 1}
\addplot[blue]{x};
\addlegendentry{Case 2}
\end{groupplot}
\end{tikzpicture}
\caption{Anticipation of the behaviours of the parsers, grammar 2}
\end{figure}
\FloatBarrier

Since the number of recursive calls of the naive parser with that grammar is linear it should be very effective compared to the `boolean' bottom-up one.
The naive parser should also be more effective than the `string' bottom-up parser since this one also follows a power function.

\subsubsection{Comparing the efficiency}

With that grammar there is only two cases to compare, the case in which the string matches the grammar and the case when it doesn't, except for the `string' bottom-up which has a behaviour depending on the pattern but for which those two patterns are the extrem cases.

The first used pattern is a case in which the strings match the grammar: `$a\string^ n$'.

\FloatBarrier
\begin{figure}[h]
\begin{tikzpicture}
    \begin{groupplot}[group style={group size=2 by 2},height=0.5\textwidth,width=0.5\textwidth] 
    \nextgroupplot[title=Both bottom-up counters, ylabel=iterations, legend pos = north west]
    \addplot coordinates {
        (50, 187575)
        (100, 1500150)
        (200, 12000300)
        (400, 96000600)
        (600, 324000900)
        (800, 768001200)
        (1000, 1500001500)};
    \addlegendentry{String}
    \addplot coordinates {
        (50, 41800)
        (100, 333600)
        (200, 2667200)
        (400, 21334400)
        (600, 72001600)
        (800, 170668800)
        (1000, 333336000)};
    \addlegendentry{Boolean}
    \nextgroupplot[title=Both bottom-up running time, legend pos = north west, ylabel=seconds]
    \addplot coordinates {
        (50, 0.011391)
        (100, 0.08719)
        (200, 0.668587)
        (400, 5.41536)
        (600, 17.6169)
        (800, 41.9493)
        (1000, 82.579)};
    \addlegendentry{String}
    \addplot coordinates {
        (50, 0.001673)
        (100, 0.011914)
        (200, 0.101755)
        (400, 0.816538)
        (600, 3.0517)
        (800, 8.58051)
        (1000, 18.6903)};
    \addlegendentry{Boolean}
    \nextgroupplot[title=Naive counter, xlabel=string size, ylabel=recursive calls]
    \addplot coordinates {
        (50000, 99999)
        (100000, 199999)
        (200000, 399999)
        (300000, 599999)
        (400000, 799999)};
    \nextgroupplot[title=Naive running time, xlabel=string size, ylabel=seconds]
    \addplot coordinates {
        (50000, 0.001854)
        (100000, 0.002961)
        (200000, 0.005776)
        (300000, 0.007833)
        (400000, 0.010237)};
    \end{groupplot}
\end{tikzpicture}
\caption{Both bottom-up and naive parsers behaviours, grammar 2, case 1}
\end{figure}
\FloatBarrier

The running time of the naive parser compared to the two bottom-up ones with this grammar is so small that it is needed to use different string sizes in order to get relevant results.

With this grammar and that case the `boolean' bottom-up parser is faster than the `string' version.

A limitation of the top-down parser is that it needs to allocate memory for its table, but with strings of the size that are used in this experimentation it is not possible to allocate the memory.
So even if the behaviour is theoricaly exactly the same as with the naive parser in practice with a classic computer it is not possible to parse strings that long with the top-down parser.

The running time of the naive parser is linear, which makes sense since its number of recursive calls is linear too.

An interesting thing is that even if the running time of the naive parser remains very small with big string sizes the maximum size of the recursive stack was reached for strings bigger than 410000 characters.
This can be explained by the fact that when the string size $n$ increases the algorithm goes deeper and deeper in the stack: 1 more level of depth every 2 recursive calls.
This is not an issue that we will be able to see with a pattern that does not starts with a `b', indeed in that case as explained in section 3.2.2.1, case 2, the maximum recursive depth accessed is 2, which will allow the parser to parse longer strings.

The second used pattern is a case in which the strings do not match the grammar: `$b\string^ n$'.

\FloatBarrier
\begin{figure}[h]
\begin{tikzpicture}
    \begin{groupplot}[group style={group size=2 by 2},height=0.5\textwidth,width=0.5\textwidth] 
    \nextgroupplot[title=Both bottom-up counters, ylabel=iterations, legend pos = north west]
    \addplot coordinates {
        (50, 62625)
        (100, 500250)
        (200, 4000500)
        (400, 32001000)
        (600, 108001500)
        (800, 256002000)
        (1000, 500002500)};
    \addlegendentry{String}
    \addplot coordinates {
        (50, 41800)
        (100, 333600)
        (200, 2667200)
        (400, 21334400)
        (600, 72001600)
        (800, 170668800)
        (1000, 333336000)};
    \addlegendentry{Boolean}
    \nextgroupplot[title=Both bottom-up running times, legend pos = north west, ylabel=seconds]
    \addplot coordinates {
        (50, 0.005147)
        (100, 0.043097)
        (200, 0.329631)
        (400, 2.56894)
        (600, 8.39397)
        (800, 20.0372)
        (1000, 39.8311)};
    \addlegendentry{String}
    \addplot coordinates {
        (50, 0.001673)
        (100, 0.011914)
        (200, 0.101755)
        (400, 0.816538)
        (600, 3.0517)
        (800, 8.58051)
        (1000, 18.6903)};
    \addlegendentry{Boolean}
    \nextgroupplot[title=Naive counter, xlabel=string size, ylabel=recursive calls]
    \addplot coordinates {
        (50000000, 50000000)
        (100000000, 100000000)
        (200000000, 200000000)
        (300000000, 300000000)
        (400000000, 400000000)
        (500000000, 500000000)
        (600000000, 600000000)};
    \nextgroupplot[title=Naive running time, xlabel=string size, ylabel=seconds]
    \addplot coordinates {
        (50000000, 0.262654)
        (100000000, 0.524298)
        (200000000, 1.05584)
        (300000000, 1.56371)
        (400000000, 2.08152)
        (500000000, 2.61476)
        (600000000, 3.13239)};
    \end{groupplot}
\end{tikzpicture}
\caption{Both bottom-up and naive parsers behaviour, grammar 2, case 2}
\end{figure}
\FloatBarrier

The curves of the `boolean' are the same as with the last case since no matter the pattern used that parser will always have the same behaviour with that grammar.
This case is the best case for the `string' bottom-up parser and the running time is indeed lower than with the first case, however it still takes more time than the `boolean' version.

With that case it was possible to parse a word of 600 millions character in about 3 seconds with the naive parser, which is very impressive.
The parser was maybe able to parse the word in some seconds but the longest part was in fact the time to generate the string and also the time to pass such string in parameter to some functions.
The running time is once again linear as expected.
\\
\\
With both the best and the worst case scenarios for the `string' bottom-up parser it is possible to display the area between the two curves.
The running time of that parser will always be contained in the blue area and the more `$b$'s there is in the given string the more the running time will follow a close curve from the worst case and conversely.

\FloatBarrier
\begin{figure}[h]
    \centering
    \begin{tikzpicture}
        \begin{axis}[title=`String' bottom-up running times, ylabel=seconds, xlabel=string size, legend pos = north west]
        \addplot[mark=*, green, name path=plot1] coordinates {
        (50, 0.005147)
        (100, 0.043097)
        (200, 0.329631)
        (400, 2.56894)
        (600, 8.39397)
        (800, 20.0372)
        (1000, 39.8311)};
        \addlegendentry{Best case}
        \addplot[mark=*, red, name path=plot2] coordinates {
        (50, 0.011391)
        (100, 0.08719)
        (200, 0.668587)
        (400, 5.41536)
        (600, 17.6169)
        (800, 41.9493)
        (1000, 82.579)};
        \addlegendentry{Worst case}
        \addplot[blue!30] fill between[of=plot1 and plot2];
        \addlegendentry{All cases}
        \end{axis}
    \end{tikzpicture}
    \caption{Running time of the `string' bottom-up parser, grammar 3}
\end{figure}
\FloatBarrier


\subsection{String ending with an `a'}
The grammar used in this section is the following:

\begin{align*}
    &S \to BA\\
    &A \to a\\
    &B \to BB|a|b
\end{align*}

\subsubsection{Number of iterations/recursive calls for each parser}

An interesting thing to notice is that with that grammar and a string of size $n$, no matter the used pattern, the amount of needed iterations or recursive calls will be the same for the naive, top-down and `boolean' bottom-up parsers.

The `string' bottom-up parser follows exactly the same behaviour as with the previous grammar.

By decomposing the behaviour of the naive parser one can find out that for any pattern it evolves following that sequence:
\begin{equation}\label{seq:naive}
n + (n - 1)^2
\end{equation}

It can be demonstrated like so:

\begin{itemize}
    \item[$-$] The initial call is $parse(S, 0, n)$
    \item[$-$] Then it calls $parse(B, 0, 1)$, which returns true no matter the character.
    \item[$-$] Then it calls $parse(A, 1, n)$, which returns false since A doesn't possess any \textit{non-terminals}.
    \item[$-$] Then it calls $parse(B, 0, 2)$, which returns true by doing 2 more recursive calls, $parse(B, 0, n)$ will always return true and do two more recursive calls each time it is called.
    \item[$-$] The process ends up when $parse(A, n - 1, n)$ is called, then it returns true or false depending on if the last character is an `a' or a `b'. No matter the result, and so the pattern, the algorithm will always make the same amount of recursive calls.
    \item[$-$] When it ends up the number of recursive calls is $1$ for the initial call, plus $n - 1$ for each time it has called parse(A, x, n), plus $(n - 1)^2$ for each time it has called parse(B, 0, x) and all the recursive calls it made. So the total number of recursive calls is $n + (n - 1)^2$ for the naive parser.
\end{itemize}

The top-down parser follows the exact same behaviour as the naive parser, to parse a string of size $n$ it will also need $n + (n - 1)^2$ recursive calls.

However some sub-problems will not need to be recomputed by the top-down parser, it is possible to show that, for a string of size $n$ and whatever pattern is used, it needs to solve that many sub-problems:
\begin{equation}\label{seq:top-down}
n + (n - 1)^2 - \dfrac{(n - 2) \cdot (n - 1)}{2}
\end{equation}

Here is the demonstration:

\begin{itemize}
    \item[$-$] The running process is exactly the same as for the naive one, the difference is that some results have already been computed and can be directly returned.
    \item[$-$] The amount of results already stored in the table and then asked again increases by one for each level of recursive depth when $parse(B, 0, x)$ is called.
    \item[$-$] This amount corresponds to a triangular number sequence: $\dfrac{n \cdot (n + 1)}{2}$. The formula needs to be twisted a little to match the string size $n$ and the number of asked already known results is $\dfrac{(n - 1) \cdot (n - 2)}{2}$.
    \item[$-$] Then the total amount of solved sub-problems is just the number of recursive calls minus the expression above: $n + (n - 1)^2 - \dfrac{(n - 2) \cdot (n - 1)}{2}$.
\end{itemize}

It is now possible to compare the anticipated behaviours of the naive, top-down and bottom-up parsers, using functions \ref{seq:naive} and \ref{eq:bottom-up_iterations}.

\FloatBarrier
\begin{figure}[h]
\begin{tikzpicture}
    \begin{groupplot}[group style={group size=2 by 1},height=0.5\textwidth,width=0.5\textwidth] 
    \nextgroupplot[title=`Boolean' bottom-up, xlabel=string size, ylabel=iterations]
    \addplot[red, domain=0:1000]{2 * x + ((3 / 6) * ((3 * (x^2) * (x + 1)) - (x * (x + 1) * ((2 * x) + 1))))};
    \nextgroupplot[title=Naive and top-down, xlabel=string size, ylabel=recursive calls]
    \addplot[blue, domain=0:50000]{x + (x - 1)^2};
    \end{groupplot}
\end{tikzpicture}
\caption{Anticipation of the behaviours of the parsers, grammar 3}
\end{figure}
\FloatBarrier

The `boolean' bottom-up parser follows a power function with an exponent of three as usual, but the top-down and naive parsers will follow a power function with an exponent of two with that grammar.

\subsubsection{Comparing the efficiency}

Except for the `string' bottom-up parser, with this grammar there is no particular cases, any pattern will be as long to parse as another by any of the three other parsers, the first used pattern is $a^n$.

\FloatBarrier
\begin{figure}[h]
\begin{tikzpicture}
    \begin{groupplot}[group style={group size=2 by 2},height=0.5\textwidth,width=0.5\textwidth] 
    \nextgroupplot[title=Both bottom-up counters, ylabel=iterations, legend pos = north west]
    \addplot coordinates {
        (100, 1500150)
        (200, 12000300)
        (400, 96000600)
        (600, 324000900)
        (800, 768001200)
        (1000, 1500001500)};
    \addlegendentry{String}
    \addplot coordinates {
        (100, 333600)
        (200, 2667200)
        (400, 21334400)
        (600, 72001600)
        (800, 170668800)
        (1000, 333336000)};
    \addlegendentry{Boolean}
    \nextgroupplot[title=Both bottom-up running times, ylabel=seconds, legend pos = north west]
    \addplot coordinates {
        (100, 0.087481)
        (200, 0.652142)
        (400, 5.16778)
        (600, 16.9815)
        (800, 40.5165)
        (1000, 79.5284)};
    \addlegendentry{String}
    \addplot coordinates {
        (100, 0.012412)
        (200, 0.102093)
        (400, 0.82625)
        (600, 3.08987)
        (800, 8.56199)
        (1000, 18.7261)};
    \addlegendentry{Boolean}
    \nextgroupplot[title=Top-down/Naive counters, ylabel=recursive calls, xlabel=string size, legend pos = north west]
    \addplot coordinates {
        (100, 9901)
        (1000, 999001)
        (3000, 8997001)
        (5000, 24995001)
        (7000, 48993001)
        (9000, 80991001)};
    \nextgroupplot[title=Top-down/Naive running times, ylabel=seconds, legend pos=north west, xlabel=string size]
    \addplot coordinates {
        (100, 0.000158)
        (1000, 0.016837)
        (3000, 0.176553)
        (5000, 0.50593)
        (7000, 1.05917)
        (9000, 1.69652)};
    \addlegendentry{TD parsing}
    \addplot coordinates {
        (100, 0.00027)
        (1000, 0.020982)
        (3000, 0.207965)
        (5000, 0.595206)
        (7000, 1.23223)
        (9000, 1.99075)};
    \addlegendentry{TD total}
    \addplot coordinates {
        (100, 0.000091)
        (1000, 0.004109)
        (3000, 0.031383)
        (5000, 0.089245)
        (7000, 0.17303)
        (9000, 0.29419)};
    \addlegendentry{TD init}
    \addplot coordinates {
        (100, 0.000173)
        (1000, 0.01458)
        (3000, 0.134243)
        (5000, 0.374054)
        (7000, 0.776454)
        (9000, 1.17694)};
    \addlegendentry{Naive}
    \end{groupplot}
\end{tikzpicture}
\caption{The 4 parsers behaviours, grammar 3}
\end{figure}
\FloatBarrier

With strings of sizes greater than $9,000$ the top-down parser could not allocate the table so the used strings did not exceed that size.

The same string sizes are used for the top-down and naive parsers, it is possible to see on the graphs above that despite the fact that the naive parser and the top-down parsers need the same amount of recursive calls the naive parser has a lower running time.

The only parser that needs a various number of iterations is the `string' bottom-up parser, the pattern `$a\string^ n$' was as previously its worst case, its best case with that grammar is the pattern `$b\string^ n$' as previously.
It is not a suprise to find out that the `string' bottom-up parser is slower than the `boolean' version with that grammar, since it follows the same behaviour as before.

Next are the results obtained with the `string' bottom-up parser, using the last results for the `boolean' bottom-up parser as reference.

\FloatBarrier
\begin{figure}[h]
\begin{tikzpicture}
    \begin{groupplot}[group style={group size=2 by 1},height=0.5\textwidth,width=0.5\textwidth] 
    \nextgroupplot[title=Both bottom-up counters, ylabel=iterations, xlabel=string size, legend pos = north west]
    \addplot coordinates {
        (100, 500250)
        (200, 4000500)
        (400, 32001000)
        (600, 108001500)
        (800, 256002000)
        (1000, 500002500)};
    \addlegendentry{String}
    \addplot coordinates {
        (100, 333600)
        (200, 2667200)
        (400, 21334400)
        (600, 72001600)
        (800, 170668800)
        (1000, 333336000)};
    \addlegendentry{Boolean}
    \nextgroupplot[title=Both bottom-up running times, legend pos = north west, xlabel=string size, ylabel=seconds]
    \addplot coordinates {
        (100, 0.042859)
        (200, 0.322342)
        (400, 2.60902)
        (600, 8.56177)
        (800, 20.5762)
        (1000, 40.6679)};
    \addlegendentry{String}
    \addplot coordinates {
        (100, 0.012412)
        (200, 0.102093)
        (400, 0.82625)
        (600, 3.08987)
        (800, 8.56199)
        (1000, 18.7261)};
    \addlegendentry{Boolean}
    \end{groupplot}
\end{tikzpicture}
\caption{The `string' bottom-up parser worst case, grammar 3}
\end{figure}
\FloatBarrier

As with the previous grammar the best case of the `string' bottom-up parser is once again slower than the `boolean' version.
The running time follows exactly the same curve as the number of iterations for the `string' bottom-up parser.

It would be possible to plot the area containing all the possible cases of the `string' bottom-up parser once again but it would be the exact same figure as before since the running time of the parser for each pattern remains the same with both grammars.

