The grammar is the following:

\begin{align*} 
    &S \to SS|LA|LR\\
    &A \to SR\\
    &L \to (\\
    &R \to )
\end{align*}

With this grammar it is possible to demonstrate that the top-down parser needs less recursive calls when it can `detect' that the string does not match the grammar very quickly.

\subsubsection{Number of iterations/recursive calls for each parsers}

With this grammar two interesting cases will be considered:

\begin{enumerate}
    \item Any pattern starting with a right parenthese:
    $$
    patterns = 
    \begin{cases}
        )\string^ n\\
        ) \text{ } (\string^ n - 1\\
        )\string^ n - 1 \text{ } (\\
    \end{cases}
    $$
    Many other patterns corresponding to the same case could be found, as long as the first character is a right parenthese.
    This case is interesting because it is the case in which the top-down parser needs to solve the less sub-problems in order to parse the strings.
    
    For example if it parses a string of size $n$ based on one of the patterns above it will need to solve $2n - 1$ sub-problems to give the result.
    That information, while interesting, is not relevant to anticipate the running time of the parser since even if we know how many sub-problems the parser will solve it can need a very different amount of recursive calls.
    
    \item The other interesting case is the pattern `$(\string^ n$', that pattern is the worst that can be generated with this grammar for the top-down parser.

    With that pattern the top-down parser will need to solve $n^2 + \lfloor \dfrac{n}{2} \rfloor$ sub-problems to parse a string of size $n$.
\end{enumerate}

Of course as demonstrated in the section 2.4 with equation \ref{eq:bottom-up_iterations} the `boolean' bottom-up parser always needs the same amount of iterations for a string of size $n$ and a given grammar:

\begin{itemize}
    \item[$-$] $n = 500$
    \item[$-$] $gt = 2$
    \item[$-$] $gnt = 4$
\end{itemize}

$$
500 \cdot 2 + \dfrac{4}{6} \cdot (3 \cdot 500^2 \cdot (500 + 1) - 500 \cdot (500 + 1) \cdot (2 \cdot 500 + 1)) = \text{83,334,000 iterations}
$$

Which is indeed the number of iterations obtained when running the code.

The `string' bottom-up parser will also follow the same behaviour for any string pattern.

It is now possible to represent the anticiped behaviour that the `boolean' bottom-up parser will follow for both cases.

\FloatBarrier
\begin{figure}[h]
\centering
\begin{tikzpicture}
\begin{groupplot}[group style={group size=1 by 1},height=0.5\textwidth,width=0.5\textwidth, domain=0:1000]
    \nextgroupplot[title=`Boolean' bottom-up for both cases, ylabel=iterations, xlabel=string size]
    \addplot[red]{2 * x + ((4 / 6) * ((3 * (x^2) * (x + 1)) - (x * (x + 1) * ((2 * x) + 1))))};
\end{groupplot}
\end{tikzpicture}
\caption{Anticipation of the `boolean' bottom-up parser, grammar 1}
\end{figure}
\FloatBarrier

\subsubsection{Comparing the efficiency}

The first experimentation uses the case in which the strings are starting with a right patenthese: $)\string^ n$.

\FloatBarrier
\begin{figure}[h]
\begin{tikzpicture}
    \begin{groupplot}[group style={group size=2 by 2},height=0.5\textwidth,width=0.5\textwidth] 
    \nextgroupplot[title=Both bottom-up counters, ylabel=iterations, legend pos=north west]
    \addplot coordinates {
        (50, 21121)
        (100, 167246)
        (200, 1334496)
        (400, 10668996)
        (600, 36003496)
        (800, 85337996)
        (1000, 166672496)};
    \addlegendentry{String}
    \addplot coordinates {
        (50, 83400)
        (100, 666800)
        (200, 5333600)
        (400, 42667200)
        (600, 144000800)
        (800, 341334400)
        (1000, 666668000)};
    \addlegendentry{Boolean}
    \nextgroupplot[title=Both bottom-up running times, ylabel=seconds, legend pos=north west]
    \addplot coordinates {
        (50, 0.000571)
        (100, 0.003931)
        (200, 0.031064)
        (400, 0.354137)
        (600, 1.25128)
        (800, 3.00095)
        (1000, 6.0186)};
    \addlegendentry{String}
    \addplot coordinates {
        (50, 0.002598)
        (100, 0.019514)
        (200, 0.142831)
        (400, 1.17313)
        (600, 4.05125)
        (800, 10.2203)
        (1000, 20.8444)};
    \addlegendentry{Boolean}
    \nextgroupplot[title=Top-down counter, xlabel=string size, ylabel=recursive calls]
    \addplot coordinates {
        (50, 3676)
        (100, 14851)
        (200, 59701)
        (400, 239401)
        (600, 539101)
        (800, 958801)
        (1000, 1498501)};
    \nextgroupplot[title=Top-down running time, xlabel=string size, ylabel=seconds, legend pos=north west]
    \addplot coordinates {
        (50, 0.000117)
        (100, 0.000302)
        (200, 0.000982)
        (400, 0.003857)
        (600, 0.007999)
        (800, 0.015024)
        (1000, 0.022071)};
    \addlegendentry{Total}
    \addplot coordinates {
        (50, 0.000048)
        (100, 0.000097)
        (200, 0.000234)
        (400, 0.000962)
        (600, 0.00241)
        (800, 0.003652)
        (1000, 0.005607)};
    \addlegendentry{Initialization}
    \addplot coordinates {
        (50, 0.000046)
        (100, 0.000195)
        (200, 0.000733)
        (400, 0.002882)
        (600, 0.006131)
        (800, 0.011)
        (1000, 0.017119)};
    \addlegendentry{Parsing}
    \end{groupplot}
\end{tikzpicture}
\caption{Both bottom-up and top-down parsers behaviours, grammar 1, case 1}
\end{figure}
\FloatBarrier

The `boolean' bottom-up parser follows exactly the predicted behaviour, the number of iterations going up following the equation \ref{eq:bottom-up_iterations} and its running time follows exactly the same curve since for each iteration the parser does exactly the same amount of loops.
For this grammar it is slower than the `string' version.

Here the top-down parser is faster than both bottom-up parsers.
To get a more precise result the initialization and the parsing process have been timed apart from each other.
We can see that the initialization time scales up a power function.
The parsing time seems to follow the curve of the recursive calls counter.
\\
\\
It is very easy to check the complexity of the two bottom-up parsers here by resolving for both an equation of the form $y = a \cdot x^3$ by replacing $x$ and $y$ by the coordinates of a point of their graphs.

Those are respectively the theorical expressions of the `boolean' and `string' bottom-up parsers:

\begin{align*}
    &y = 2.0844 \cdot 10^{-8} \cdot x^3 &y = 6.0186 \cdot 10^{-9} \cdot x^3
\end{align*}

\FloatBarrier
\begin{figure}[h]
\begin{tikzpicture}
    \begin{groupplot}[group style={group size=2 by 1},height=0.5\textwidth,width=0.5\textwidth] 
    \nextgroupplot[title=`Boolean' bottom-up, xlabel=string size, ylabel=seconds, legend pos=north west]
    \addplot[domain=0:1000, samples=100, line width=1.5pt, green] {2.0844*(10^(-8))*(x^3)};
    \addlegendentry{Theory}
    \addplot[only marks, mark=*, mark size=2pt] coordinates {
        (50, 0.002598)
        (100, 0.019514)
        (200, 0.142831)
        (400, 1.17313)
        (600, 4.05125)
        (800, 10.2203)
        (1000, 20.8444)};
    \addlegendentry{Real}
    \nextgroupplot[title=`String' bottom-up, xlabel=string size, ylabel=seconds, legend pos=north west]
    \addplot[domain=0:1000, samples=100, line width=1.5pt, green] {6.0186*(10^(-9))*(x^3)};
    \addlegendentry{Theory}
    \addplot[only marks, mark=*, mark size=2pt] coordinates {
        (50, 0.000571)
        (100, 0.003931)
        (200, 0.031064)
        (400, 0.354137)
        (600, 1.25128)
        (800, 3.00095)
        (1000, 6.0186)};
    \addlegendentry{Real}
    \end{groupplot}
\end{tikzpicture}
\caption{Checking theorical fit, both bottom-up parsers}
\end{figure}
\FloatBarrier

The two functions fit almost perfectly to the points which confirms the $O(n^3)$ complexity for both bottom-up parsers.
\\
\\
The naive parser is very slow with that grammar, but it is possible to check its complexity too by resolving the equation $y = a \cdot 3^x$.

\begin{align*}
    &y = 2.146422 \cdot 10^{-10} \cdot 3^x
\end{align*}

\FloatBarrier
\begin{figure}[h]
\centering
\begin{tikzpicture}
    \begin{groupplot}[group style={group size=1 by 1},height=0.5\textwidth,width=0.5\textwidth] 
    \nextgroupplot[title=naive, xlabel=string size, ylabel=seconds, legend pos=north west]
    \addplot[domain=10:23, samples=100, line width=1.5pt, green] {2.146422*(10^(-10))*(3^x)};
    \addlegendentry{Theory}
    \addplot[only marks, mark=*, mark size=2pt] coordinates {
        (10, 0.000236)
        (15, 0.017009)
        (18, 0.248498)
        (20, 1.39282)
        (21, 3.36746)
        (22, 7.98121)
        (23, 20.2071)};
    \addlegendentry{Real}
    \end{groupplot}
\end{tikzpicture}
\caption{Checking theorical fit, naive parser}
\end{figure}
\FloatBarrier

The theorical curve fits perfectly the running time of the naive parser which proves that the complexity of the naive parser is $O(3^n)$, it also shows that with that grammar this parser is very bad compared to the others.
\\
\\
The second pattern is the pattern only constituated by left parentheses: `$(\string^ n$', what makes it interesting is the fact that it is the worst case that we can generate with this grammar for the top-down parser.
The bottom-up parsers results will not be displayed in that case since they have the same behaviour as with the first case.

\FloatBarrier
\begin{figure}[h]
\begin{tikzpicture}
    \begin{groupplot}[group style={group size=2 by 1},height=0.5\textwidth,width=0.5\textwidth] 
    \nextgroupplot[title=Top-down counter, xlabel=string size, ylabel=recursive calls]
    \addplot coordinates {
        (50, 43526)
        (100, 340801)
        (200, 2696601)
        (400, 21453201)
        (600, 72269801)
        (800, 171146401)
        (1000, 334083001)};
    \nextgroupplot[title=Top-down running time, xlabel=string size, ylabel=seconds, legend pos=north west]
    \addplot coordinates {
        (50, 0.000605)
        (100, 0.004426)
        (200, 0.031748)
        (400, 0.252762)
        (600, 0.819545)
        (800, 1.88128)
        (1000, 3.71316)};
    \addlegendentry{Total}
    \addplot coordinates {
        (50, 0.000033)
        (100, 0.000119)
        (200, 0.000244)
        (400, 0.00081)
        (600, 0.001904)
        (800, 0.003088)
        (1000, 0.005164)};
    \addlegendentry{Initialization}
    \addplot coordinates {
        (50, 0.000561)
        (100, 0.004297)
        (200, 0.031488)
        (400, 0.251933)
        (600, 0.817614)
        (800, 1.87816)
        (1000, 3.70797)};
    \addlegendentry{Parsing}
    \end{groupplot}
\end{tikzpicture}
\caption{Top-down parser behaviour, grammar 1, case 2}
\end{figure}
\FloatBarrier

As we can see on the graph above in that case the initialization time of the top-down parser is negligible, the parsing time is the one that matters.
The parsing time follows exactly the same curve as the number of recursive calls.
Despite the fact that this case is the worst one for the top-down parser its running time remains lower than the other parsers.

We can conclude that for this grammar the top-down parser is more efficient than both bottom-up algorithms.
\\
\\
Here we can verify the complexity of the top-down parser by resolving the same equation as for the two bottom-up parsers.

\begin{align*}
    &y = 3.10797 \cdot 10^{-9} \cdot x^3
\end{align*}

\FloatBarrier
\begin{figure}[h]
\centering
\begin{tikzpicture}
    \begin{groupplot}[group style={group size=1 by 1},height=0.5\textwidth,width=0.5\textwidth] 
    \nextgroupplot[title=top-down, xlabel=string size, ylabel=seconds, legend pos=north west]
    \addplot[domain=0:1000, samples=100, line width=1.5pt, green] {3.70797*(10^(-9))*(x^3)};
    \addlegendentry{Theory}
    \addplot[only marks, mark=*, mark size=2pt] coordinates {
        (50, 0.000561)
        (100, 0.004297)
        (200, 0.031488)
        (400, 0.251933)
        (600, 0.817614)
        (800, 1.87816)
        (1000, 3.70797)};
    \addlegendentry{Real}
    \end{groupplot}
\end{tikzpicture}
\caption{Checking theorical fit, grammar 1, case 2}
\end{figure}
\FloatBarrier

The curve fits perfectly the points of the top-down parser which proves that the complexity of the algorithm is $O(n^3)$.

