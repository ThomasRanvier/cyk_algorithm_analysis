The grammar used in this section is the following:

\begin{align*}
    &S \to BA\\
    &A \to a\\
    &B \to BB|a|b
\end{align*}

\subsubsection{Number of iterations/recursive calls for each parser}

An interesting thing to notice is that with that grammar and a string of size $n$, no matter the used pattern, the amount of needed iterations or recursive calls will be the same for the naive, top-down and `boolean' bottom-up parsers.

The `string' bottom-up parser follows exactly the same behaviour as with the previous grammar.

By decomposing the behaviour of the naive parser one can find out that for any pattern it evolves following that sequence:
\begin{equation}\label{seq:naive}
n + (n - 1)^2
\end{equation}

It can be demonstrated like so:

\begin{itemize}
    \item[$-$] The initial call is $parse(S, 0, n)$
    \item[$-$] Then it calls $parse(B, 0, 1)$, which returns true no matter the character.
    \item[$-$] Then it calls $parse(A, 1, n)$, which returns false since A doesn't possess any \textit{non-terminals}.
    \item[$-$] Then it calls $parse(B, 0, 2)$, which returns true by doing 2 more recursive calls, $parse(B, 0, n)$ will always return true and do two more recursive calls each time it is called.
    \item[$-$] The process ends up when $parse(A, n - 1, n)$ is called, then it returns true or false depending on if the last character is an `a' or a `b'. No matter the result, and so the pattern, the algorithm will always make the same amount of recursive calls.
    \item[$-$] When it ends up the number of recursive calls is $1$ for the initial call, plus $n - 1$ for each time it has called parse(A, x, n), plus $(n - 1)^2$ for each time it has called parse(B, 0, x) and all the recursive calls it made. So the total number of recursive calls is $n + (n - 1)^2$ for the naive parser.
\end{itemize}

The top-down parser follows the exact same behaviour as the naive parser, to parse a string of size $n$ it will also need $n + (n - 1)^2$ recursive calls.

However some sub-problems will not need to be recomputed by the top-down parser, it is possible to show that, for a string of size $n$ and whatever pattern is used, it needs to solve that many sub-problems:
\begin{equation}\label{seq:top-down}
n + (n - 1)^2 - \dfrac{(n - 2) \cdot (n - 1)}{2}
\end{equation}

Here is the demonstration:

\begin{itemize}
    \item[$-$] The running process is exactly the same as for the naive one, the difference is that some results have already been computed and can be directly returned.
    \item[$-$] The amount of results already stored in the table and then asked again increases by one for each level of recursive depth when $parse(B, 0, x)$ is called.
    \item[$-$] This amount corresponds to a triangular number sequence: $\dfrac{n \cdot (n + 1)}{2}$. The formula needs to be twisted a little to match the string size $n$ and the number of asked already known results is $\dfrac{(n - 1) \cdot (n - 2)}{2}$.
    \item[$-$] Then the total amount of solved sub-problems is just the number of recursive calls minus the expression above: $n + (n - 1)^2 - \dfrac{(n - 2) \cdot (n - 1)}{2}$.
\end{itemize}

It is now possible to compare the anticipated behaviours of the naive, top-down and bottom-up parsers, using functions \ref{seq:naive} and \ref{eq:bottom-up_iterations}.

\FloatBarrier
\begin{figure}[h]
\begin{tikzpicture}
    \begin{groupplot}[group style={group size=2 by 1},height=0.5\textwidth,width=0.5\textwidth] 
    \nextgroupplot[title=`Boolean' bottom-up, xlabel=string size, ylabel=iterations]
    \addplot[red, domain=0:1000]{2 * x + ((3 / 6) * ((3 * (x^2) * (x + 1)) - (x * (x + 1) * ((2 * x) + 1))))};
    \nextgroupplot[title=Naive and top-down, xlabel=string size, ylabel=recursive calls]
    \addplot[blue, domain=0:50000]{x + (x - 1)^2};
    \end{groupplot}
\end{tikzpicture}
\caption{Anticipation of the behaviours of the parsers, grammar 3}
\end{figure}
\FloatBarrier

The `boolean' bottom-up parser follows a power function with an exponent of three as usual, but the top-down and naive parsers will follow a power function with an exponent of two with that grammar.

\subsubsection{Comparing the efficiency}

Except for the `string' bottom-up parser, with this grammar there is no particular cases, any pattern will be as long to parse as another by any of the three other parsers, the first used pattern is $a^n$.

\FloatBarrier
\begin{figure}[h]
\begin{tikzpicture}
    \begin{groupplot}[group style={group size=2 by 2},height=0.5\textwidth,width=0.5\textwidth] 
    \nextgroupplot[title=Both bottom-up counters, ylabel=iterations, legend pos = north west]
    \addplot coordinates {
        (100, 1500150)
        (200, 12000300)
        (400, 96000600)
        (600, 324000900)
        (800, 768001200)
        (1000, 1500001500)};
    \addlegendentry{String}
    \addplot coordinates {
        (100, 333600)
        (200, 2667200)
        (400, 21334400)
        (600, 72001600)
        (800, 170668800)
        (1000, 333336000)};
    \addlegendentry{Boolean}
    \nextgroupplot[title=Both bottom-up running times, ylabel=seconds, legend pos = north west]
    \addplot coordinates {
        (100, 0.087481)
        (200, 0.652142)
        (400, 5.16778)
        (600, 16.9815)
        (800, 40.5165)
        (1000, 79.5284)};
    \addlegendentry{String}
    \addplot coordinates {
        (100, 0.012412)
        (200, 0.102093)
        (400, 0.82625)
        (600, 3.08987)
        (800, 8.56199)
        (1000, 18.7261)};
    \addlegendentry{Boolean}
    \nextgroupplot[title=Top-down/Naive counters, ylabel=recursive calls, xlabel=string size, legend pos = north west]
    \addplot coordinates {
        (100, 9901)
        (1000, 999001)
        (3000, 8997001)
        (5000, 24995001)
        (7000, 48993001)
        (9000, 80991001)};
    \nextgroupplot[title=Top-down/Naive running times, ylabel=seconds, legend pos=north west, xlabel=string size]
    \addplot coordinates {
        (100, 0.000158)
        (1000, 0.016837)
        (3000, 0.176553)
        (5000, 0.50593)
        (7000, 1.05917)
        (9000, 1.69652)};
    \addlegendentry{TD parsing}
    \addplot coordinates {
        (100, 0.00027)
        (1000, 0.020982)
        (3000, 0.207965)
        (5000, 0.595206)
        (7000, 1.23223)
        (9000, 1.99075)};
    \addlegendentry{TD total}
    \addplot coordinates {
        (100, 0.000091)
        (1000, 0.004109)
        (3000, 0.031383)
        (5000, 0.089245)
        (7000, 0.17303)
        (9000, 0.29419)};
    \addlegendentry{TD init}
    \addplot coordinates {
        (100, 0.000173)
        (1000, 0.01458)
        (3000, 0.134243)
        (5000, 0.374054)
        (7000, 0.776454)
        (9000, 1.17694)};
    \addlegendentry{Naive}
    \end{groupplot}
\end{tikzpicture}
\caption{The 4 parsers behaviours, grammar 3}
\end{figure}
\FloatBarrier

With strings of sizes greater than $9,000$ the top-down parser could not allocate the table so the used strings did not exceed that size.

The same string sizes are used for the top-down and naive parsers, it is possible to see on the graphs above that despite the fact that the naive parser and the top-down parsers need the same amount of recursive calls the naive parser has a lower running time.

The only parser that needs a various number of iterations is the `string' bottom-up parser, the pattern `$a\string^ n$' was as previously its worst case, its best case with that grammar is the pattern `$b\string^ n$' as previously.
It is not a suprise to find out that the `string' bottom-up parser is slower than the `boolean' version with that grammar, since it follows the same behaviour as before.

Next are the results obtained with the `string' bottom-up parser, using the last results for the `boolean' bottom-up parser as reference.

\FloatBarrier
\begin{figure}[h]
\begin{tikzpicture}
    \begin{groupplot}[group style={group size=2 by 1},height=0.5\textwidth,width=0.5\textwidth] 
    \nextgroupplot[title=Both bottom-up counters, ylabel=iterations, xlabel=string size, legend pos = north west]
    \addplot coordinates {
        (100, 500250)
        (200, 4000500)
        (400, 32001000)
        (600, 108001500)
        (800, 256002000)
        (1000, 500002500)};
    \addlegendentry{String}
    \addplot coordinates {
        (100, 333600)
        (200, 2667200)
        (400, 21334400)
        (600, 72001600)
        (800, 170668800)
        (1000, 333336000)};
    \addlegendentry{Boolean}
    \nextgroupplot[title=Both bottom-up running times, legend pos = north west, xlabel=string size, ylabel=seconds]
    \addplot coordinates {
        (100, 0.042859)
        (200, 0.322342)
        (400, 2.60902)
        (600, 8.56177)
        (800, 20.5762)
        (1000, 40.6679)};
    \addlegendentry{String}
    \addplot coordinates {
        (100, 0.012412)
        (200, 0.102093)
        (400, 0.82625)
        (600, 3.08987)
        (800, 8.56199)
        (1000, 18.7261)};
    \addlegendentry{Boolean}
    \end{groupplot}
\end{tikzpicture}
\caption{The `string' bottom-up parser worst case, grammar 3}
\end{figure}
\FloatBarrier

As with the previous grammar the best case of the `string' bottom-up parser is once again slower than the `boolean' version.
The running time follows exactly the same curve as the number of iterations for the `string' bottom-up parser.

It would be possible to plot the area containing all the possible cases of the `string' bottom-up parser once again but it would be the exact same figure as before since the running time of the parser for each pattern remains the same with both grammars.
