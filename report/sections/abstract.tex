This paper presents six different implementations of the Cocke-Younger-Kasami algorithm (CYK algorithm for short).
The first one is a naive version, it uses the Divide and conquer approach and has a complexity of $O(3^n)$.
The other implementations make use of Dynamic programming, there are two versions using bottom-up approach and the others use the top-down method, they all have a complexity of $O(n^3)$.

The four first parsers are classic implementations of the CYK algorithm using different programming methods, the last two have some specialities.
After the presentation of the implementations and the analysis of the different complexities a chapter is focused on experimentations with the four first parsers using diverse context-free grammars in the Chomsky normal-form.

The implementation of a parser able to make use of linear grammars with a running time of $O(n^3)$ and based on the top-down implementation is presented.
Experimentations and comparisons with the results of the four previous parsers are made.

A parser that automatically corrects the input string in order to make it match the given grammar (when possible) in $O(n^3)$ is presented.
This parser is based on the top-down initial implementation, it suggests to the user a correction of the initial string and indicates the number of character modifications and character deletions that were needed to reach total correctness.
